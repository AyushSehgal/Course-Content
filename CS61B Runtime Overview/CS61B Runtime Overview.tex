\documentclass[11pt,letterpaper]{article}
\usepackage[lmargin=1in,rmargin=1in,tmargin=1in,bmargin=1in]{geometry}
\usepackage{worksheet}
\usepackage{listings}
\usepackage{color}

\definecolor{dkgreen}{rgb}{0,0.6,0}
\definecolor{gray}{rgb}{0.5,0.5,0.5}
\definecolor{mauve}{rgb}{0.58,0,0.82}

\lstset{frame=tb,
  language=Java,
  aboveskip=3mm,
  belowskip=3mm,
  showstringspaces=false,
  columns=flexible,
  basicstyle={\small\ttfamily},
  numbers=none,
  numberstyle=\tiny\color{gray},
  keywordstyle=\color{blue},
  commentstyle=\color{dkgreen},
  stringstyle=\color{mauve},
  breaklines=true,
  breakatwhitespace=true,
  tabsize=3
}

% -------------------
% Content
% -------------------
\begin{document}
\worksheet{Runtime Practice}


% Question 1
\problem Orders of Growth. 
\begin{enumerate}[(a)]
    \item Simplify using Big-O Notation: $N^{2} + NlogN + 1000N + 6$
    \pspace
    \pspace
    \pspace
    \pspace
    \pspace
    \item Calculate the following limit:
    $\lim_{n \to \infty} \frac{n^{2}}{2^{n}}$
    \pspace
    \pspace
    \pspace
    \pspace
    \pspace
    \item What does the result of the above limit imply about the exponential and polynomial runtimes. 
    \textit{Hint: Is one always greater (grows faster) than the other?}
    \pspace
    \pspace
    \pspace
    \pspace
    \pspace
    \item True or False: If function f has $\mathcal{O}(n)$ and $\Omega(1)$, its tight bound is always found by taking the average of n and 1 giving: $\Theta(n)$
    \pspace
    \pspace
    \pspace
    \pspace
    \pspace
\end{enumerate}
\pspace
\newpage


% Question 2
\problem Iteration
\\ 
What is the runtime of the following functions?
\begin{lstlisting}
    public static void loopingMore(int[] array) {
        int n = array.length;
        for (int i = 0; i < 3*n*n*n; i++) {
            System.out.println('I love you');
        }
    }
\end{lstlisting}
\pspace
\pspace
\pspace
\pspace
\begin{lstlisting}
    public static void doubleLoopingHalf(int n) {
        for (int i = 0; i < n; i++) {
            for (int j = n; j > 0; j = j / 2) {
                System.out.println('I love you');
            }
        }
    }
\end{lstlisting}
\pspace
\pspace
\pspace
\pspace

\begin{lstlisting}
    public static void weirdLooping(int n) {
        for (int i = 0; i < n; i++) {
            int num = Math.pow(2, i + 1) - 1;
            for (int j = 0; j < num; j++) {
                System.out.println('I love you');
            }
        }
    }
\end{lstlisting}
\pspace
\pspace
\pspace
\pspace
\pspace
\newpage


% Question 3
\problem Recursion
\\
\textit{Practice using the Work Per Layer Summation Formula}
\begin{lstlisting}
    public static void mergeSort(int[] arr, int l, int r) {
        if (l < r) {
            // Compute Middle Index
            int m = (l + r) / 2
            
            mergeSort(arr, l, m);
            mergeSort(arr, m+1, r);
            
            merge(arr, l, m, r); // Runs in linear time 
        }
    }
\end{lstlisting}
\pspace
\pspace
\pspace
\pspace

\begin{lstlisting}
    public static void thinTree(int n) {
        if (N <= 1) {return;}
        else {
            thinTree(2);
            thinTree(n / 2);
        }
        
    }
\end{lstlisting}
\pspace
\pspace
\pspace
\pspace

\begin{lstlisting}
    public static void splitter(int n) {
        if (N == 1) {return;}
        else {
            int i = 0;
            while (i < n) {
                i++
                System.out.println(i);
            }
            return splitter(n / 3) + splitter(2n / 3);
        }
        
    }
\end{lstlisting}
\newpage




% Question 4
\problem Mutual Recursion 
\\
\textit{The following problem has been adapted from Prof. Hug's Sp15 Midterm 2}
\\
\\
State the runtime of each function starting with f1.
\begin{lstlisting}
    public static void f1(int n) {
        for (int i = 0; i < 2*n; i++) {
            System.out.println('Welcome');
        }
    }
\end{lstlisting}

\begin{lstlisting}
    public static void f2(int n) {
        if (n == 0) { return; }
        f2(n / 3);
        f1(n);
        f2(n / 3);
        f1(n);
        f2(n / 3);
    }
\end{lstlisting}

\begin{lstlisting}
    public static void f3(int n) {
        if (n == 0) { return; }
        f3(n - 1);
        f1(16);
        f3(n - 1);
    }
\end{lstlisting}

\begin{lstlisting}
    public static void f4(int n) {
        if (n == 0) { return; }
        f4(n - 1);
        f1(16);
        f1(n);
        f4(n - 1);
    }
\end{lstlisting}

\begin{lstlisting}
    public static void f5(int n, int m) {
        if (m <= 0) {
            return;
        } else {
            for (int i = 0; i < n; i++) {
                f5(n, m-1);
            }
        }
    }
\end{lstlisting}



\end{document}