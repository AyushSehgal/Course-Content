\documentclass{article}
\usepackage[utf8]{inputenc}

\title{Asymptotic Analysis Guide}
\author{Ayush Sehgal}
\date{CS61B: Data Structures}

\begin{document}

\maketitle

\section{Introduction}
The goal of courses like CS61A, CS88, and E7 is to teach you how to solve problems programmatically. The aim in this class is to not only learn how to solve problems but how to solve them in an efficient manner. So we are not looking for just any old solution but for the "best" or most feasible one given certain conditions. But how do we determine what is the "best" solution? Asymptotics! \\
Asymptotics will help you understand why solution A may be better than solution B and will allow you to justify that numerically. \\
There are two metrics we can use when we discuss asymptotics: 
\begin{enumerate}
    \item Time Complexity
    \item Space Complexity
\end{enumerate}
Note! It is not necessarily the case that the faster running algorithm is always better, a really fast algorithm might waste a lot of space for instance, or may not be possible to implement given the computers/tools we have today. 

\end{document}
